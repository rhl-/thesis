% commands.tex
% Afra Zomorodian
% August 20, 2009
\usepackage{lineno}

%\usepackage[usenames,dvipsnames]{color}
%\usepackage{subfig}
\usepackage{amscd}
\usepackage{amsmath}
\usepackage{amsthm}
\usepackage{amssymb}
\usepackage{ifthen}
\usepackage{eucal} % for mathcal
\usepackage{url}
\usepackage{clrscode}
\usepackage{multirow}
\usepackage{graphicx}
\usepackage{proof}
\usepackage{times}
\usepackage{amsthm}
\usepackage{tikz}
\usepackage{tikz-cd}
\usepackage{epsfig}
\usepackage{subcaption}
\usepackage{shortlst}
%\usepackage{rotating}
%\usepackage{enumitem}
%\setlist{nolistsep}
\usepackage{multicol}
\usepackage{prelimdraft}
\usepackage{siunitx}
\setcounter{page}{0}

%% units are in inches
%\setlength{\unitlength}{1in}
%
%% Useful macros for graphics
%% centerfig
%\newcommand{\centerfig}[2]{%
%\centerline{\includegraphics[#2]{figs/#1}}
%}
%
%\newcommand{\boxedcenterfig}[2]{%
%%\setlength{\fboxsep}{0pt}
%\centerline{\fbox{\includegraphics[#2]{figs/#1}}}
%}
%
%\setlength{\fboxsep}{1pt}
%
%%% Define a new style for the package that will use a smaller font.
%\makeatletter
%\def\url@smallurlstyle{%
%  \@ifundefined{selectfont}{\def\UrlFont{\sf}}{\def\UrlFont{\small\ttfamily}}}
%\makeatother
%% Now actually use the newly defined style.
%\urlstyle{smallurl}
%
%% subimg
%\newcommand{\subimg}[5][-1]{%
%\subfigure[#5]{%
%  \label{#4}
%  \ifthenelse {\equal{#1}{-1}}
%    {\includegraphics[#3]{figs/#2}}
%    {\begin{minipage}[b]{#1\columnwidth}{%
%      \centerline{\includegraphics[#3]{figs/#2}}}
%    \end{minipage}
%    }
%}}
%
%% subimgnt
%\newcommand{\subimgnt}[3][-1]{%
%\subfigure{%
%  \ifthenelse {\equal{#1}{-1}}
%    {\includegraphics[#3]{figs/#2}}
%    {\begin{minipage}[b]{#1\columnwidth}{%
%      \centerline{\includegraphics[#3]{figs/#2}}}
%    \end{minipage}
%    }
%}}

% mathbb
\newcommand{\R}{\ensuremath{\mathbb{R}}}
\newcommand{\NN}{\ensuremath{\mathbb{N}}}
\newcommand{\Q}{\ensuremath{\mathbb{Q}}}
\newcommand{\Z}{\ensuremath{\mathbb{Z}}}
\newcommand{\F}{\ensuremath{\mathbb{F}}}
\newcommand{\emax}{\ensuremath{\hat{\epsilon}}}
\newcommand{\vr}{\ensuremath{\mathcal{V}}}
\newcommand{\vrs}{\ensuremath{\vr_{\epsilon}(S)}}
\newcommand{\es}{\ensuremath{E_{\epsilon}(S)}}
\newcommand{\vs}{\ensuremath{V_{\epsilon}(S)}}
\newcommand{\vrms}{\ensuremath{\vr_{\emax}(S)}}
\newcommand{\ges}{\ensuremath{G_{\epsilon}(S)}}
\newcommand{\gemax}{\ensuremath{G_{\emax}(S)}}
\newcommand{\weight}{\ensuremath{\omega}}
\newcommand{\gweight}{\ensuremath{w}}
\DeclareMathOperator{\dist}{d}
\newcommand{\cplusplus}{C\raisebox{0.5ex}{\small ++}}
\newcommand{\Filt}[1]{\ensuremath{\leq_{#1}}}

% names
\newcommand{\cech}{\v{C}ech} % not in math mode
\newcommand{\acm}{ACM}
\newcommand{\ieee}{IEEE}
\newcommand{\siggraph}{SIGGRAPH}
\newcommand{\etlsh}{\textsc{E\raisebox{0.7ex}{\scriptsize 2}LSH}}
\newcommand{\ann}{\textsc{Ann}}
\newcommand{\cgal}{\textsc{Cgal}}
\newcommand{\plex}{\textsc{Plex}}

%%complexity classes
\newcommand{\complexity}[1]{\textsc{#1}}
\newcommand{\NPH}{\complexity{NP-Hard}}
\newcommand{\NP}{\complexity{NP}}
\newcommand{\NPC}{\complexity{NP-Complete}}
\newcommand{\APX}{\complexity{APX}}
\newcommand{\APXC}{\complexity{APX-Complete}}

%%decision problems
\newcommand{\ablp}{\textsc{$\alpha$-Balanced-Minimum-Blowup}}
\newcommand{\dablp}{\textsc{Dual-$\alpha$-Balanced-Minimum-Blowup}}
\newcommand{\aedge}{\textsc{$\alpha$-Balanced-Edge-Separator}}
\newcommand{\avertex}{\textsc{$\alpha$-Subgraph-Balanced-Vertex-Separator}}
\newcommand{\aBMB}{\textsc{$\alpha$-Bmb}}
\newcommand{\aBES}{\textsc{$\alpha$-Bes}}
\newcommand{\aBVS}{\textsc{$\alpha$-Bvs}}
\newcommand{\YES}{\textsc{Yes}}

\newcommand{\scitem}{\textsc{\normalfont\item}}

% algorithms
\newcommand{\allpairs}{\proc{All-Pairs}}
\newcommand{\scan}{\proc{Scan}}
\newcommand{\kdexact}{\proc{Kd-Exact}}
\newcommand{\kdapprox}{\proc{Kd-Approx}}
\newcommand{\bbdexact}{\proc{BBD-Exact}}
\newcommand{\bbdapprox}{\proc{BBD-Approx}}
\newcommand{\witness}{\proc{Witness}}
\newcommand{\Foreach}{\textbf{foreach} }

% expansion
\newcommand{\inductive}{\proc{Inductive-VR}}
\newcommand{\incremental}{\proc{Incremental-VR}}
\newcommand{\maximal}{\proc{Maximal-VR}}
\newcommand{\lowerneighbors}{\proc{Lower-Nbrs}}
\newcommand{\addcofaces}{\proc{Add-Cofaces}}
\newcommand{\ikgx}{\proc{IK-GX}}
\newcommand{\generatecombinations}{\proc{Generate-Combinations}}
\newcommand{\computeweightfunction}{\proc{Compute-Weights}}
\newcommand{\computeweight}{\proc{Weight}}
\newcommand{\ceil}[1]{\left \lceil#1\right \rceil}
\newcommand{\floor}[1]{\lfloor#1\rfloor}

% data sets
\newcommand{\bunny}{{\textsf{B}}}
\newcommand{\sphere}{\textsf{S}}
\newcommand{\clique}{\textsf{C}}
\newcommand{\multiblob}{\textsf{M}}
\newcommand{\blobs}{\multiblob}
\newcommand{\gnp}{\textsf{G}}

% Erdos-Renyi
\newcommand{\Erdos}{Erd\H{o}s}
\newcommand{\Renyi}{R{\'e}nyi}
\newcommand{\almosta}{a.a.\@}

\renewcommand{\Im}[1]{\operatorname{Im}{#1}}
\newcommand{\Ker}[1]{\operatorname{Ker}{#1}}
%\newcommand{\rank}[1]{\operatorname{rank}{#1}}
\newcommand{\Dim}[1]{\operatorname{dim}{#1}}
\newcommand{\bd}{\ensuremath{\partial}}
\DeclareMathOperator{\im}{im}
\DeclareMathOperator{\size}{size}
\newcommand{\betti}{\ensuremath{\beta}}
\newcommand{\tensor}{\otimes}
\DeclareMathOperator{\rank}{rank}
\newcommand{\Bd}[1]{\partial_{#1}}
\newcommand{\Cl}[1]{\operatorname{Cl}{#1}}
\newcommand{\bigslant}[2]{{\raisebox{.2em}{$#1$}\left/\raisebox{-.2em}{$#2$}\right.}}
\newcommand{\card}[1]{|#1|}
\newcommand{\mv}{Mayer-Vietoris }
\newcommand{\mvb}{\mv blowup complex}
\newcommand{\K}{K}
\renewcommand{\L}{L}
\newcommand{\C}{U}
\newcommand{\N}{N}
\newcommand{\M}{M}
\newcommand{\factor}{\card{\K^{\C}}/\card{\K}}
\newcommand{\ratio}{\factor}
\newcommand{\Parfor}{\kw{parallel for}}
\newtheorem{theorem}{Theorem}
\newtheorem{lemma}{Lemma}
%\newtheorem{corollary}[theorem]{Corollary}
\newtheorem{corollary}{Corollary}[lemma]
\theoremstyle{definition}
\newtheorem{example}{Example}[subsection]
\newtheorem{fact}{Fact}[subsection]
\newtheorem{definition}{Definition}[section]

%%%%%%%%%%
%Afra's Colors
%%%%%%%%%
\definecolor{afra}{RGB}{198,168,208}
\definecolor{afrablue}{RGB}{143,166,215}
\definecolor{afragreen}{RGB}{182,215,112}
\definecolor{afrapurple}{RGB}{218,177,239}
\definecolor{darkgray}{gray}{0.3}
\definecolor{afrapurplelight}{RGB}{198,168,208}
\definecolor{afrabluedark}{RGB}{75,113,191}
\definecolor{afrabluelight}{RGB}{143,166,215}
\definecolor{afragreendark}{RGB}{154,191,75}
\definecolor{afragreenlight}{RGB}{182,215,112}
\definecolor{afrapurpledark}{RGB}{162,75,191}

\usepackage{xcolor}
\usepackage{pgfplots}
\usepackage{pgfplotstable}
\usepackage{filecontents}
\usetikzlibrary{calc, fit, shadows, arrows, positioning, patterns, external, matrix}
% if these lines are commented out, the box draws.
\pgfdeclarelayer{edges}
\pgfdeclarelayer{quadcell}
\pgfsetlayers{quadcell,edges,main}
\pgfplotsset{compat=newest}

%%%%%%%%%%%
%%BEGIN TIKZ TODO LOGIC
%%%%%%%%%%%%%%%
\makeatletter \newcommand \listoftodos{\section*{Todo list} \@starttoc{tdo}}
\newcommand\l@todo[2]
  {\par\noindent \textit{#2}, \parbox{10cm}{#1}\par} \makeatother

\newcommand{\todo}[1]{
% Add to todo list
\addcontentsline{tdo}{todo}{\protect{#1}}
%
\begin{tikzpicture}[remember picture, baseline=-0.75ex]
    \node [coordinate] (inText) {};
\end{tikzpicture}
%
% Make the margin par
    \begin{tikzpicture}[remember picture]
        \definecolor{orange}{rgb}{1,0.5,0}
 
        \draw node[draw=black, fill=orange, text width = 5cm] (inNote)
                 {#1};
    \end{tikzpicture}
%
%\begin{tikzpicture}[remember picture, overlay]
%    \draw[draw = orange, thick]
%        ([yshift=-0.2cm] inText)
%            -| ([xshift=-0.2cm] inNote.west)
%            -| (inNote.west);
%\end{tikzpicture}
%
}
\graphicspath{{figs/},{figs/speedup-figs/}}

\def\lbl#1#2{\begingroup
	    #2%
	    \def\@currentlabel{#2}%
	    \phantomsection\label{#1}\endgroup
	}
