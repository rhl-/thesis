\chapter{Mayer Vietoris \& Homology}
Any parallel computation of homology would require a decomposition of the
space into pieces. So far we discussed a way of computing homology from a 
nested sequence of inclusions. While this procedure is straightforward in theory, it can suffer it practice
because it potentially requires communication between all pairs of processors. To address this
we aim to decompose our input space input space into pieces using a \emph{cover}. Again in this setting, 
there is classical theory which explains how and when one can reconstruct the homology of a space
from the elements of a cover and there intersections. In the case of a cover by two sets this is referred to as
the \mv long exact sequence. When there are more than two sets in the cover, 
there is too much algebraic data to form a long exact sequence. 
In the case of a cover with three or more elements, the \mv spectral sequence expresses 
the relationship between the homology of these pieces of a space and the homology of the space itself.
This makes the \mv spectral sequence a natural gadget to study when developing algorithms for parallel homology~\cite{hatcher}. 

In this chapter we design and implement a framework for the parallel computation of field homology using the \mv principle. In the next section we describe the Mayer Vietoris Spectral Sequence. We then discuss the a topological analogue, on the input to this spectral sequence the \mvb{}.  
The \mvb{} is used by Zomorodian and Carlsson to compute localized homology~\cite{zc-lh-08}.  The procedures discussed in this section 
are a framework, in the sense that we provide an algorithm, which takes as input a cover. Any cover may be used with this algorithm,
and the results will be identical, with the only observable difference being the running time and space usage. 

We then present a simplistic model for the problem of finding covers which will lend themselves to parallelism and show that finding such covers is \NPH{}.
We describe an algorithm for producing covers with a simple structure and bounded overlap based on graph partitions. This approach allows
us to avoid the explicit construction of the blowup complex saving time and space. We implement our algorithms for multicore computers, 
and demonstrate their efficacy with a suite of experiments.  For example, we achieve roughly an 8$\times$ 
speedup of the homology computations on a 3-dimensional complex with about 10 million simplices using 11 cores.

\section{Mayer Vietoris Spectral Sequence}
\begin{figure}[h]
\centering
\begin{tikzcd}[scale=1,
execute at end scope={\begin{pgfonlayer}{edges}
\node[xshift=-1em,yshift=.5em] (c2) at (e20.north west) {\footnotesize$C_2\left(\K^{\C}\right)$};
\node[xshift=-1em,yshift=.5em] (c1) at (e10.north west) {\footnotesize$C_1\left(\K^{\C}\right)$};
\node[xshift=-1em,yshift=.5em] (c0) at (e00.north west) {\footnotesize$C_0\left(\K^{\C}\right)$};
\draw[opacity=.5,line width=7mm,line cap=round,color=afrablue] (e20.center) to (e02.center); 
\draw[opacity=.5,line width=7mm,line cap=round,color=afragreen] (e10.center) to (e01.center); 
\draw[opacity=.5,line width=7mm,line cap=round,color=afrapurple] (e00.center) to (e00.center);
\end{pgfonlayer} 
 }]
|[alias=e30]|  \vdots \arrow{d}{\partial_{\K}}& |[alias=e31]| \vdots \arrow{d}{\partial_{\K}}& |[alias=e32]| \vdots \arrow{d}{\partial_{\K}} \\
|[alias=e20]|E^0_{2,0}  \arrow{d}{\partial_{\K}}                     &  |[alias=e21]| E^0_{2,1} \arrow{l}{\partial_{\N}} \arrow{d}{\partial_{\K}}&|[alias=e22]|  E^0_{2,2} \arrow{l}{\partial_{\N}}  \arrow{d}{\partial_{\K}}& \ldots \arrow{l}{\partial_{\N}} \\
|[alias=e10]|E^0_{1,0} \arrow{d}{\partial_{\K}}                     & |[alias=e11]| E^0_{1,1} \arrow{l}{\partial_{\N}}  \arrow{d}{\partial_{\K}}& |[alias=e12]| E^0_{1,2} \arrow{l}{\partial_{\N}} \arrow{d}{\partial_{\K}}& \ldots \arrow{l}{\partial_{\N}} \\
|[alias=e00]|E^0_{0,0}                                      & |[alias=e01]| E^0_{0,1} \arrow{l}{\partial_{\N}}                & |[alias=e02]| E^0_{0,2} \arrow{l}{\partial_{\N}}                   & \ldots \arrow{l}{\partial_{\N}} 
\end{tikzcd}
\caption{The $E_0$ page of the Mayer Vietoris Double Complex. The term $E^0_{i,j}$ is the vector space generated by the $i$-dimensional cells in a $j$-fold intersection. Horizontal maps correspond to taking the nerve boundary, vertical maps correspond to the boundary of the base complex. The cells of dimension $p$ in the blowup complex are found on the anti-diagonal defined by $i+j = p$.}
\end{figure}
Recall, that a spectral sequence is a sequence of chain complexes $(E^r, d^r)$ called pages, where the terms of $E^{r+1}$ is composed of the homology on the previous page. This is an example of a spectral sequence of a double complex. In the context of a double complex, we have more algebraic data on each page. In particular, If we have a simplicial complex $\K$ with cover $\C$, then we have $E^r = \bigoplus_{p,q} E^r_{p,q}$ where $E^0_{p,q}$ is the vector space generated by cells of the form $\sigma \otimes \tau$ where $\sigma \in \K$ and $\tau \in N(\C)$ and $\dim{\sigma} = p$ and $\dim{\tau} = q$. 
We may define a differential $d^0: E^0_{p,q} \rightarrow E^0_{p-1,q}$ given by $d^0: \sigma \otimes \tau \rightarrow \partial_\K(\sigma) \otimes \tau$, making $(E^0, d^0)$ the first iterate in our spectral sequence. As usual, the homology of the $d^0$ differential provides the terms of the $E^1$ page. Unsurprisingly, the next differential, $d^1$ will be defined similarly, in terms of the right factor and $\partial_N$, the boundary on the nerve of the cover. This provides us with the $E^2$ page of the spectral sequence. Observe that by swapping the roles of $p$ and $q$ we produce a different spectral sequence.
Before describing this map, observe that we have two different filtrations of vector spaces which we may construct from this data: 
\begin{enumerate}
\item $\tilde{E}_k = \bigoplus_{p, q \leq k} E^0_{p,q}$
\item $\tilde{E}_k = \bigoplus_{p \leq k, q} E^0_{p,q}$ 
\end{enumerate}
Each of these filtrations of vector spaces gives rise to a spectral sequence as defined in the previous chapter, where $d^0 + (-1)^{p+q}d^1$ and $d^1 + (-1)^{p+q}d^0$ is the boundary operator. One can show that these spectral sequences have the same $E_\infty$ page~\cite{mcleary}. However, it is not difficult to produce an example where these filtrations produce distinct persistent homology modules. Consider for instance, the space and associated cover in Figure~\ref{fig:vignette}. 

Returning the spectral sequence of the double complex, The $d^r$ differential has the following signature $d^r_{p,q}: E^r_{p,q} \rightarrow E^r_{p+r-1, q-r}$. The map $d^r_{p,q}$ is defined as the restriction of the differential $r$ to elements in $E^r_{p,q}$. 

It is initially difficult to produce example spaces where even $d^2_{p,q}$ is nontrivial. We provide some here. In a later chapter we will revisit these examples in the context of our algorithms. To do this we introduce the Hopf map.

\begin{example}
\end{example}

