\begin{figure*}
\centering
\subfigure[Space and Cover]{
	\label{fig:space-n-cover}
	\def\svgwidth{1.25in}
	\input{figs/blowup_parts.pdf_tex}
}
\hfill
\subfigure[Local pieces of the blowup complex.]{
	\label{fig:local-pieces}
	\def\svgwidth{1.1in}
	\input{figs/local_pieces.pdf_tex}
}
\hfill
\subfigure[The blowup complex.]{
	\label{fig:blowup}
	\def\svgwidth{1.1in}
	\input{figs/blowup_complex.pdf_tex}
}
%\hfill
%\subfigure[Persistence Barcode. Colors represent homology groups computed
%	   on each piece of the blowup complex. Stacked colors represent 
%  parallelism]{
%	\label{fig:barcode}
%	\def\svgwidth{1.1in}
%	\input{figs/barcode.pdf_tex}
%}
\caption{Our approach. We are given a space equipped with a 
	 cover~\subref{fig:space-n-cover}, the former represented by a path with
         four vertices and three edges and the latter represented by ovals.  
	 First, at time $(t = 0)$ we blowup up the space into 
	 local pieces~\subref{fig:local-pieces}, each local piece is a copy of 
	 the corresponding cover set, then, at $(t = 1)$ we glue together 
	 duplicated simplices by adding in the blowup cells, rendering them 
	 homologically equivalent, which gives us the blowup 
	 complex~\subref{fig:blowup}.
}
\label{fig:vignette}
\end{figure*}
