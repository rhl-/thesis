



\chapter{Introduction}
In modern society, data is everywhere. As a result of this proliferation of information, an emphasis has been placed on methods for studying data.
Applied topology is a new field of mathematics interested in quantifying the shape of data. When one considers applied mathematics, or
data analysis, usually traditional fields such as partial differential equations or statistics come to mind, not topology. However, it turns out that
there is a myriad of data problems for which the question of shape becomes paramount. Examples of this are found in biology, robotics, ...., blah blah blah.

Modern applied topology perhaps began with, Frozini, who introduced the notion of size functions to analyze the shapes of manifolds. 
Then, ELZ studied PH of alpha shapes in $R^3$. Carlsson-Zomorodian generalized this. Then Zigzags. Carlsson Desilva,  
then multi-d Carlsson, Singh Zomorodian and zig-zags.

\begin{figure}
PUT SOMETHING HERE
\caption{Data sampled from an annulus}
\cite{fig:annulus}
\end{figure}
Persistent homology models the space from which data is sampled from. Figure~\ref{fig:annulus-pts} at this dataset X on the plane. By itself the topology of this set of points is not very interesting. However, if you squint your eyes enough this collection of points looks like an annulus. To model this suppose that for some real number $\epsilon > 0$ we considered the union of $\epsilon$ balls in the plane. If we choose $\epsilon = .5$ then we have described a shape shown in Figure~\ref{fig:annulus-top}. Of course if we chose this parameter too small or too large the shape we construct would be too under or over connected and not model the connectivity of the annulus well. Persistent topology addresses this issue by studying how the shape of these model spaces change as we vary this parameter. 



\subsection{Background}
\subsection{To see utility from persistent homology, we must be able to compute it efficiently.}