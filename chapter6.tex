\chapter{Conclusion}
In this work we presented two procedures for computing the [persistent] homology of a topological space by first computing homology on a collection of sub complexes and there intersections, and then gluing this information together using the \mv principle. We presented theoretical results which demonstrate that the use of \mv can limit the space usage of homology computation when computing the homology of large sub complexes with small intersections. We also presented experimental results demonstrating that our algorithms are competitive with existing algorithms, both parallel and serial. We also demonstrated an explicit connection between the spectral sequence of a filtration and the persistent homology module. While not necessarily a new result, we are unaware of this statement being made explicitly anywhere else to date. In terms of future work, the spectral construction of the persistent homology module allows practitioners , to explore \emph{incomplete} invariants for the persistent homology over $\Z$ computationally. Additionally, we believe that by combining the results for computing persistent homology via \mv with many well known practical optimizations, will yield a persistence pipeline which will work on \emph{massive datasets}. Finally, in the realm of theoretical contributions, there is usually added algebraic structure on double complexes. For example a multiplicative structure, providing an graded algebra. In would be of interest to understand what these operations tell us about the persistent homology module, and about our underlying dataset.