\begin{figure*}
\begin{subfigure}[t]{.31\textwidth}
\centering
%\hspace*{-1cm}%
\begin{tikzpicture}[scale=.45, y=0.6pt, x=.75pt]
     %row 1
     \node[font=\large] at (-50, 800) {$(K, U)$};
     \draw[fill=afragreen, draw = black,  line width=2]  (47,800) circle (10pt);  
     \draw[fill=afragreen, draw = black, line width=2]  (117,800) circle (10pt);   
     \draw[fill=afragreen, draw = black, line width=2]  (190,800) circle (10pt); 
     \draw[fill=afragreen, draw = black, line width=2]  (261,800) circle (10pt); 
     \begin{pgfonlayer}{edges}
            \path[draw=black,fill=black,line width=2] (47,800) -- (117, 800);
            \path[draw=black,fill=black,line width=2] (117,800) -- (190, 800);
            \path[draw=black,fill=black,line width=2] (190,800) -- (261, 800);
      \end{pgfonlayer}      
      \draw[draw, color=afrablue, fill=none, line join=round,draw opacity=0.978,line width=2] (117, 800) ellipse (100 and 45);
      \draw[draw, color=afrapurple, fill=none, line join=round,draw opacity=0.978,line width=2] (190, 800) ellipse (100 and 45);
         %row 2 
    \node[font=\large] at (-40, 740) {$K^0$};
     \draw[fill=afragreen, draw = black, line width=2]  (47,730) circle (10pt);  
     \draw[fill=afragreen, draw = black, line width=2]  (117,730) circle (10pt);   
     \draw[fill=afragreen, draw = black, line width=2]  (190,730) circle (10pt); 
     \begin{pgfonlayer}{edges}
            \path[draw=black,fill=black,line width=2] (47,730) -- (117, 730);
            \path[draw=black,fill=black,line width=2] (117,730) -- (190, 730);
      \end{pgfonlayer}   
               %row 3
     \node[font=\large] at (-50, 680) {$K^1$};
     \draw[fill=afragreen, draw = black, line width=2]  (117,680) circle (10pt);   
     \draw[fill=afragreen, draw = black, line width=2]  (190,680) circle (10pt); 
      \draw[fill=afragreen, draw = black, line width=2]  (261,680) circle (10pt); 
         \begin{pgfonlayer}{edges}
            \path[draw=black,fill=black,line width=2] (117,680) -- (190, 680);
            \path[draw=black,fill=black,line width=2] (190,680) -- (261, 680);
      \end{pgfonlayer}   
          %row 4
\node[font=\large] at (-50, 620) {$K^{[1]}$};
     \draw[fill=afragreen, draw = black, line width=2]  (117,620) circle (10pt);   
     \draw[fill=afragreen, draw = black, line width=2]  (190,620) circle (10pt); 
         \begin{pgfonlayer}{edges}
            \path[draw=black,fill=black,line width=2] (117,620) -- (190, 620);
      \end{pgfonlayer}     
\end{tikzpicture}
\caption{Space and Cover}
\label{fig:space-n-cover}
\end{subfigure}
%\hfill
\begin{subfigure}[t]{.31\textwidth}
%	\def\svgwidth{1in}
%	\input{figs/local_pieces.pdf_tex}
\begin{tikzpicture}[scale=.45, y=0.6pt, x=.75pt]
    %disj 0
    \begin{scope}[shift={(143,80)}]
    \node[font=\large] at (300, 730) {$K^0 \times \Delta^{0}$};
     \draw[draw, color=afrablue, fill=none, line join=round,draw opacity=0.978,line width=2] (117, 730) ellipse (100 and 45);
     \draw[fill=afrablue, draw = black, line width=2]  (47,730) circle (10pt);  
     \draw[fill=afrablue, draw = black, line width=2]  (117,730) circle (10pt);   
     \draw[fill=afrablue, draw = black, line width=2]  (190,730) circle (10pt); 
     \begin{pgfonlayer}{edges}
            \path[draw=black,fill=black,line width=2] (47,730) -- (117, 730);
            \path[draw=black,fill=black,line width=2] (117,730) -- (190, 730);
      \end{pgfonlayer}   
      \end{scope}
      %disj  1
       \begin{scope}[shift={(0,0)}]
         \node[font=\large] at (380, 680) {$K^1 \times \Delta^{1}$};
       \draw[draw, color=afrapurple, fill=none, line join=round,draw opacity=0.978,line width=2] (190, 680) ellipse (100 and 45);
     \draw[fill=afrapurple, draw = black, line width=2]  (117,680) circle (10pt);   
     \draw[fill=afrapurple, draw = black, line width=2]  (190,680) circle (10pt); 
      \draw[fill=afrapurple, draw = black, line width=2]  (261,680) circle (10pt); 
         \begin{pgfonlayer}{edges}
            \path[draw=black,fill=black,line width=2] (117,680) -- (190, 680);
            \path[draw=black,fill=black,line width=2] (190,680) -- (261, 680);
      \end{pgfonlayer}   
            \end{scope}
\end{tikzpicture}
\caption{Disjoint union in the blowup complex.}
	\label{fig:local-pieces}
\end{subfigure}
%\hfill
\begin{subfigure}[t]{.31\textwidth}
\centering
\begin{tikzpicture}[scale=.45, y=0.6pt, x=.75pt]
    %disj 0
    \begin{scope}[shift={(143,80)}]
    \node[font=\large] at (300, 730) {$K^0 \times \Delta^{0}$};
     \draw[draw, color=afrablue, fill=none, line join=round,draw opacity=0.978,line width=2] (117, 730) ellipse (100 and 45);
     \draw[fill=afrablue, draw = black, line width=2]  (47,730) circle (10pt);  
     \draw[fill=afrablue, draw = black, line width=2]  (117,730) circle (10pt);   
     \draw[fill=afrablue, draw = black, line width=2]  (190,730) circle (10pt); 
     \begin{pgfonlayer}{edges}
            \path[draw=black,fill=black,line width=2] (47,730) -- (117, 730);
            \path[draw=black,fill=black,line width=2] (117,730) -- (190, 730);
      \end{pgfonlayer}   
      \end{scope}
      %disj  1
       \begin{scope}[shift={(0,0)}]
         \node[font=\large] at (380, 680) {$K^1 \times \Delta^{1}$};
       \draw[draw, color=afrapurple, fill=none, line join=round,draw opacity=0.978,line width=2] (190, 680) ellipse (100 and 45);
     \draw[fill=afrapurple, draw = black, line width=2]  (117,680) circle (10pt);   
     \draw[fill=afrapurple, draw = black, line width=2]  (190,680) circle (10pt); 
      \draw[fill=afrapurple, draw = black, line width=2]  (261,680) circle (10pt); 
         \begin{pgfonlayer}{edges}
            \path[draw=black,fill=black,line width=2] (117,680) -- (190, 680);
            \path[draw=black,fill=black,line width=2] (190,680) -- (261, 680);
      \end{pgfonlayer}   
            \end{scope}
      %blowup edges 
      \begin{scope}[shift={(143,80)}]
     \begin{pgfonlayer}{edges}
            \path[draw=black,fill=black,line width=2] (47,730) -- (47, 600);
            \path[draw=black,fill=black,line width=2] (117,730) -- (117, 600);
      \end{pgfonlayer}     
          \node[font=\large] at (280, 665) {$K^{[1]} \times \Delta^{[1]}$};
        \begin{pgfonlayer}{quadcell}
      \draw [fill=afragreen,  preaction={fill, afragreen}, pattern=north west lines, pattern color=black] (47, 600) rectangle (117, 730);
	\end{pgfonlayer}
	\end{scope}
\end{tikzpicture}
		\caption{The blowup complex.}
		\label{fig:blowup}
\end{subfigure}
\caption{Our approach. We are given a space equipped with a 
	 cover~(\protect\subref{fig:space-n-cover}), the former represented by a path with
         four vertices and three edges and the latter represented by ovals.  
	 First, at time $(t = 0)$ we blowup up the space into 
	 local pieces~(\protect\subref{fig:local-pieces}), each local piece is a copy of 
	 the corresponding cover set, then, at $(t = 1)$ we glue together 
	 duplicated simplices by adding in the blowup cells, rendering them 
	 homologically equivalent, which gives us the blowup 
	 complex~(\protect\subref{fig:blowup}).
}
\label{fig:vignette}
\end{figure*}
