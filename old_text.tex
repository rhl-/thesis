%intro
The filtration of the blowup complex has two phases,  the \emph{local} and the
 \emph{global} phase. In the local phase, the complex explodes into multiple 
pieces, representing the disjoint union of each set in the cover, as in Figure~(\ref{fig:local-pieces}). 
This means that we have potentially multiple versions of a simplex 
if it lies in an intersection of two sets in the cover. For example, since edge $bc$ falls within both sets in the 
cover in Figure~(\ref{fig:space-n-cover}), it is represented by two cells 
$bc \times 0$ and $bc \times 1$. The pieces at the local stage are disjoint, so 
we may compute the homology of the pieces in parallel. 

The global phase specifies cells that glue the different versions of the 
original simplices together, rendering them homologically equivalent.  
For example, in Figure~(\ref{fig:blowup}), the cell 
$b \times 01$ connects $b \times 0$ and $b \times 1$.  

To describe this filtration on the blowup complex, we assume that we have an 
arbitrary filtration $\Filt{\K}$ on the simplices of our input complex $\K$.  In practice, 
we often label the vertices of a complex using numbers or letters and use the lexicographic 
ordering of the vertices to generate a filtration on the complex. We use the same procedure 
with $\N(\C)$ as its vertices are numbered by definition.

Given a filtration $\Filt{\K}$ on $\K$ and $\Filt{\N(\C)}$ on $\Delta^n$ we define a partial order $\Filt{{\K^\C}}$ by 
ordering all cells in the local phase before those in the global phase. This amounts to 
comparing two cells $\sigma \times \Delta^M$ and $\tau \times \Delta^N$ by  
comparing the second factor according to $\Filt{\N(\C)}$. We may complete this partial order to a filtration by then comparing the 
first factor according to $\Filt{\K}$. 
\begin{example}
Figure~(\ref{fig:blowup}) has the following filtration: 
\begin{linenomath*}
\begin{equation*}
(\overbrace{a \times 0, b \times 0 ,c \times 0,  ab \times 0, bc \times 0}^
{\textrm{Local Piece \#0 } (t=0)},
\overbrace{b \times 1 , c \times 1, d \times 1, bc \times 1, cd \times 1}^
{\textrm{Local Piece \#1 }(t=0)},
\overbrace{b \times 01, c \times 01, bc \times 01}^
{\textrm{Global Piece } (t=1)}).
\end{equation*}
\end{linenomath*}
\end{example}